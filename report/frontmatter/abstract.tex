\chapter*{Abstract}
\addcontentsline{toc}{chapter}{Abstract}
\makeatletter\@mkboth{}{Abstract}\makeatother

\subsubsection*{English}

This report documents the design and implementation of a wireless communication system for a miniaturised satellite called a \textit{PocketQube}. The PocketQube standard is a set of specifications which aim to make it easier for people to design small satellite modules that can easily integrate with each other. The standard is relatively new, however, and few designs have been published for a PocketQube communication system.

The design in this project includes both a tracking ground station, as well as a PocketQube module. The LoRa-based ground station was designed to mechanically track the satellite system using a pre-defined GPS path, as well as wirelessly recieved GPS data. The system was then implemented, and the communication link was tested up to 50 km line-of-sight and found to be reliable at a baud rate of 9600 bps. The GPS tracking was also tested and proved to be a robust method. Finally, the ground station was tested with data received from a third party radiosonde as a backup using a software-defined radio. This report documents the design choices and constraints, analyses the results of the system, and suggests improvements for future projects.

\selectlanguage{english}