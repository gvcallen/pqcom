\chapter*{Abstract}
\addcontentsline{toc}{chapter}{Abstract}
\makeatletter\@mkboth{}{Abstract}\makeatother

This report documents the design and implementation of a wireless communication system for a miniaturised satellite called a \textit{PocketQube}. The PocketQube standard is a set of specifications which aim to make it easier for people to design small satellite modules that can easily integrate with each other. The standard is relatively new, however, and few designs have been published for a full PocketQube communication system that includes a ground station.

The design in this project includes both a tracking ground station, as well as a PocketQube module. The LoRa-based ground station was designed to mechanically track the satellite system using a pre-defined GPS path, as well as using satellite-received GPS data. The ground station was also designed to receive telemetry from existing radiosondes in the meteorological band. The system was then implemented, and the LoRa communication link was successfully tested up to 122 km line-of-sight at a baud rate of 9600 bps. This report documents the design choices made and constraints faced; discusses the methods uses during implementation; interprets the test results; and suggests improvements for future projects.

\selectlanguage{english}