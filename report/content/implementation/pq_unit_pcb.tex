\graphicspath{{./figures}}

\section{PocketQube Unit PCB}
Unfortunately, not enough prototypes were ready for an end-project balloon launch. The power section was therefore implemented using large but readily available converters/regulators. In a final launch, this would either be replaced by the EPS, or a low drop-out voltage regulator. The PQ unit PCB was ordered through an external supplier. In order to begin initial development and testing, an Arduino Nano breadboard prototype was created, as shown in Appendix \ref{sec:appendix_pq}. Since the Nano has the same MCU as the designed PCB, the developed software could then simply be flashed onto the PQ unit. The final module is shown in Figures \ref{fig:pqUnitPCB} and \ref{fig:pqUnitPower}. As noted in the appendix errata, the crystal's loading capacitors were mistakingly left out. The MCU's internal 8 MHz clock was therefore used, however was badly calibrated, and resulted in GPS software serial communication initially not working. This was fixed by setting the OSCCAL register appropriately. The board was first bootloaded with an Arduino Nano, and then the programmer circuit in Figure \ref{fig:pqUnitPCB} was developed to flash the board via UART. The programmer consists of a CH340 USB-to-serial module, a reset RC filter, and a $\SI{3.3}{V}$ to $\SI{5}{V}$ level converter (since the PocketQube PCB is designed to run at $\SI{3.3}{V}$). The board and dipole antenna was mounted onto the 2000 mAh LiPo battery. The power section consists of a LiPo charger, a boost converter module, and a $\SI{3.3}{V}$ linear voltage regulator circuit.