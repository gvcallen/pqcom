\graphicspath{{./figures}}

\section{PocketQube}

The PocketQube standard is a fairly new set of protocols and specifications defining a modularized nano-satellite system. The term \textit{modularized} in this context refers to the ability for different ``modules" or PocketQube units, each with their own set functionality, to be connected to a common \textit{backplane} (as in Figure \ref{fig:pq_backplane}) and integrated seamlessly. \textit{Integration} here refers to both the mechanical spacing of each module, as well as the electronic communication between the modules.

As an example, a PocketQube enclosure could contain three units: a communication module, a sensor pack, and a battery system. These modules can then be connected onto a PCB backplane via headers, placed inside a single enclosure (such as that in Figure \ref{fig:pq_enclosure}) and either launched into orbit, or released as a balloon satellite payload.