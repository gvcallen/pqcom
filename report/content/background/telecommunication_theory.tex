\section{Telecommunication Theory}\label{sec:telecommunication_theory}
A review of the communication techniques used for wireless links is conducted below. Since it is already planned to use modular components in the hardware design stage, and most modules handle error correction, modulation, demodulation and more with little effort, only a brief overview is given.

\subsection{Modulation Techniques}
The \textit{modulation} technique used over a communication link essentially refers to the method used to encode analog or digital information into an electrical voltage. ASK, FSK, PSK, and QAM are the most common techniques \cite{site-satelliteModulationOverview} \cite{site-satelliteModulationComparison}:
\begin{itemize}
    \item \textit{Amplitude Shift Keying} (ASK). This technique modulates the amplitude of the voltage to represent various bit levels.
    \item \textit{Frequency Shift Keying} (FSK). This method changes the frequency of the signal to represent various signal levels. Commonly, \textit{Gaussian-FSK} (GFSK) is used, which implements a smoothing filter. This decreased the bandwidth for similar throughput, making it comparable to the following techniques.
    \item \textit{Phase Shift Keying} (PSK). This is often seen as a variantion of FSK, where the phase of the signal is varied instead of the frequency. It required complex synchronisation, and has lower spectral efficiency compared to QAM.
    \item \textit{Quadrature Amplitude Modulation} (QAM). This method modulates both the amplitude and phase of a signal, allowing for a large number of signal levels. It has very high spectral efficiency, and is commonly used for higher cost systems.
\end{itemize}

\textit{loRa} ("Long-Range") is a relatively new technique which offers makes use of frequency-varying "chirps" to modulation the incoming signal. It has been seen to drastically increase range capacity, while sacrificing bit rate. Its practical use has also been demonstrated in an existing PocketQube application in \cite{design-FOSSASATLink}.