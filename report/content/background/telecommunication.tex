\section{Telecommunication}\label{sec:telecommunication_theory}

\subsection{Modulation Techniques}
The \textit{modulation} technique used over a digital communication link refers to the method used to represent bits as electrical signals. Common techniques are compared below.

\begin{table}[!htb]
  \centering
  \renewcommand{\arraystretch}{1.2}
  \hspace*{-0.5cm}
  \begin{tabular}{ |p{3cm}|p{2cm}|p{11cm}| }
  \hline
  \textbf{Name}                & \textbf{Bit Variation}    & \textbf{Comments} \\ \hline
   Amplitude Shift Keying (ASK)
   & Amplitudes
   & Poor noise immunity
   \\ \hline
   Frequency Shift Keying (FSK)
   & Frequencies
   & Simple to implement. Can be filtered for decreased bandwidth, and is then known as \textit{Gaussian-FSK} (GFSK).
   \\ \hline
   Phase Shift \mbox{Keying} (PSK)
   & Phase \mbox{Angles}
   & Similar to FSK or QAM. Lower spectral efficiency than QAM.
   \\ \hline
   Quadrature Amplitude Modulation (QAM)
   & Amplitude-Phase Pairs
   & Very high spectral efficiency. Commonly used in higher cost, high throughput systems.
   \\ \hline
   LoRa (``Long-Range") Modulation
   & Frequency-varying ``Chirps"
   & Can drastically increase range, at the expense of greater bandwidth requirements \cite{datasheet-SX1278}. Practically demonstrated in an existing CubeSat system \cite{design-FOSSASATLink}.
   \\ \hline
  \end{tabular}
  \caption{Comparison of Modulation Techniques}
  \label{tab:modulationTechniques}
\end{table}
\vspace*{-\baselineskip}

\subsection{Link Budget}\label{sec:link_budget}
For any communication system, a \textit{link budget} should be estimated, with the goal of taking into account transmitter power, antenna characteristics, receiver sensitivity, and any attenuation involved. A number of the attenuation affects include: cable loss; amplifier-antenna mismatch; free-space path losses; absorption losses due to clouds and rain; \textit{polarization} mismatch due to antennas not being aligned; \textit{scintillation} effects due to changes in the air's refractive index; and the effect of varying \textit{elevation angles} \cite{design-satelliteLinkBudget}. It is recommended to include a minimum \textit{fade margin} of at least 3 dB for low frequency links \cite{paper-linkMargin}; 10 dB for links that require more up-time \cite{design-linkBudgetDesign}; and 20 dB for critical links.

\subsection{Protocols}
A few link-layer and network-layer protocols already exist that facilitate nano-satellite applications. 
The CubeSat Space Protocol (CSP) \cite{standard-csp} is a fully-fledged network layer protocol for CubeSat embedded systems. It allows for all devices in the satellite ecosystem (satellite modules, the ground station, a control computer etc.) to communicate ``directly" with each other via \textit{packet forwarding}. It relies on various link and network -layer protocols, including KISS, I2C and CCSDS (\textit{Consultative Committee for Space Data Systems}).