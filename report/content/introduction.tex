\chapter{Introduction}

This project aims to design and implement a wireless communication system for a miniaturised satellite called a PocketQube (PQ). The PocketQube standard was created to define physical and electronic requirements for so-called "nano satellites". The goal of this is to allow for easy integration of various sub-modules into one physical enclosure. One common use-case of these satellites is to collect sensory information from the atmosphere. These can either be placed into orbit, are attached to a \textit{high-altitude balloons} (e.g. a large, helium balloon). In this project, such a balloon will been provided by the department, and a communication system for this satellite balloon will be designed.

This project consists of the design of three sub-systems. The communication system to be designed will involve both a tracking ground station (GS), as well as a PQ 'unit'. The general idea is that the GS will mechanically track the PQ while it communicates with it, enabling realtime data transmission and telemetry. An existing two-axis antenna mount has been provided for the GS, which a newly-designed PCB will be placed into. The GS will mechanically the balloon, allowing bi-directional wireless communication. The PQ unit should conform to the PQ standard and fit inside a provided housing. The third sub-system is the integration of a proprietary Radiosonde (atmospheric telemetry device) into the newly-designed communication system.

Literature will be consulted in order to investigate the various design approaches and decisions. Since the project focuses on system design and integration, a number of components and sub-modules will ultimately be combined to form the final system. Different electronic components and their specifications will be compared based on gathered project requirements. This will require trade-offs to be made, since the system design is limited in time, cost, form-factor and several other factors. This report documents the trade-offs and decisions made, with the goal of being of use to future designers of similar systems.