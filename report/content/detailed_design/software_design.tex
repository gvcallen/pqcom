\graphicspath{{./figures}}

\section{Software Design}

It was decided to follow a modular approach when designing the software, since many systems components were either re-usable, such as the GPS or two stepper motors, or logically containable. A shared library was designed for C++ to be re-used between the PocketQube and ground station code. Composition was used in code instead of inheritance, to follow the component-based design. The following describes the various classes planned to be used, as well as their expected functionality:

\begin{table}[!htb]
  \centering
  \renewcommand{\arraystretch}{1.2}
  \begin{tabular}{ |c|c| }
  \hline
  \textbf{Function}        & \textbf{Description}    \\
  \hline
    stepForward(numSteps)  & Blocking and non-blocking options \\
    saveZeroPosition() & Used for calibration \\
    getPosition() & Used for open-loop feedback \\
    setSpeed() & Sets the delay between steps \\
    setCurrentMultiplier() & Set the amount of current \\
  \hline
  \end{tabular}
  \caption{\textbf{StepperMotor} Class}
  \label{tab:stepperMotorUML}
\end{table}

\begin{table}[!htb]
  \centering
  \renewcommand{\arraystretch}{1.2}
  \begin{tabular}{ |c|c| }
  \hline
  \textbf{Function}        & \textbf{Description}    \\
  \hline
    calibrate()                         & Calbrate the mount \\
    setAzimuthalElevation(az, el)       & Set the azimuthal and elevation angles \\
    setBoresight(boresightVec)          & Set the boresight pointing vector \\
  \hline
  \end{tabular}
  \caption{\textbf{Mount} Class}
  \label{tab:mountUML}
\end{table}

\begin{table}[!htb]
  \centering
  \renewcommand{\arraystretch}{1.2}
  \begin{tabular}{ |c|c| }
  \hline
  \textbf{Function}        & \textbf{Description}    \\
  \hline
    startTransmit(message)              & Transmit non-blocking \\
    startReceive()                      & Receive non-blocking \\
    getRssi()                           & Get signal strength \\
    getSnr()                            & Get signal-to-noise ratio \\
  \hline
  \end{tabular}
  \caption{\textbf{Radio} Class}
  \label{tab:radioUML}
\end{table}

\begin{table}[!htb]
  \centering
  \renewcommand{\arraystretch}{1.2}
  \begin{tabular}{ |c|c| }
  \hline
  \textbf{Function}        & \textbf{Description}    \\
  \hline
    getLocation()              & Return latitude, longitude and altitude \\
    getTime()                  & Return seconds since epoch \\
    getDate()                  & Return date \\
  \hline
  \end{tabular}
  \caption{\textbf{GPS} Class}
  \label{tab:gpsUML}
\end{table}

\begin{table}[!htb]
  \centering
  \renewcommand{\arraystretch}{1.2}
  \begin{tabular}{ |c|c| }
  \hline
  \textbf{Function}        & \textbf{Description}    \\
  \hline
    getLinearAccel()              & Return the current linear acceleration \\
    getRotationalAccel()           & Return the current rotational acceleration \\
    getOrientation()                  & Return the current orientation rotation matrix \\
  \hline
  \end{tabular}
  \caption{\textbf{IMU} Class}
  \label{tab:imuUML}
\end{table}

\begin{table}[!htb]
  \centering
  \renewcommand{\arraystretch}{1.2}
  \begin{tabular}{ |c|c| }
  \hline
  \textbf{Function}             & \textbf{Description}    \\
  \hline
    calibrate()                 & Calibrate the entire GS \\
    addEstimatedLocation(loc)      & Add an estimated input GPS location for open-loop tracking \\
    addKnownLocation(loc)          & Add a known GPS location for closed-loop tracking \\
  \hline
  \end{tabular}
  \caption{\textbf{GroundStation} Class}
  \label{tab:groundStationUML}
\end{table}

\newpage
Various smaller modules will also be needed with limited functionality, such as:
\begin{enumerate}
    \item \textbf{Math} module, for vector calculations, WGS84 GPS coordinate projection, angle radian-degrees conversions
    \item \textbf{Magnetic declination} lookup table, for determining the magnetic declination at given instant given a GPS co-ordinate
    \item \textbf{PqTnc} class, which provides the serial connection between the GS and a computer.
\end{enumerate}