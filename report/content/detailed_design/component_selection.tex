\graphicspath{{./figures}}

\section{Component Selection}
The first step in the detailed design process is to select components for the various sub-systems. Components for both the ground station and PQ unit will be selected in one step, due to the overlapping research involved. Through-hole components and breakout boards will mostly be used for the ground station, and surface mount components for the PQ unit.

The ground station has a number components which will be used without modification. The existing stepper motors draw 0.50 A and allow for up to $\SI{0.5}{N \cdot m}$ of torque \cite{datasheet-4118}. The ground station also includes a small zero-sensing circuit for the azimuthal direction, which integrates a \textit{H22A} optical switch \cite{datasheet-H22A1} and a low-pass filter.

\subsection{GPS}\label{sec:components_gps}
The same GPS will be selected for both the ground station and PQ unit to simplify development. A small one degree pointing accuracy will be designed for, since it is an order of magnitude less than the expected beamwidth (discussed in Section \ref{sec:tracking}). A distance of \SI{120}{km} therefore results in a required accuracy of $2 \times 120 000 \times \pi \times \frac{1}{360} \approx \SI{2}{km}$. The NEO line of GPS modules from \textit{u-blox} (e.g. the \textit{NEO-6M}) are very commonly used, and are quoted to have a 2.5 m accuracy, however there was limited availability of these modules from local suppliers at the time of ordering. Other modules with similiar specifications were therefore considered, such as the \textit{ATGM336H} and the \textit{ATGM332D}. Since both of these advertise 2.5 m accuracy, active antenna support, and current consumption less than 100 mA, the ATGM332D-5N31 was chosen for its lower price.

\subsection{IMU}
To automatically determine absolute ground station orientation, the ground station (either the base or the platform) may include an inertial measurement unit (IMU). Alternatively, a \textit{dead-reckoning} fix can be done by ensuring the ground station is flat and manually positioning it to face magnetic north using a compass or a magnetometer. To provide maximum flexibility in the implementation stage, a 9-axis IMU, which includes both an accelerometer and a magnetometer, will be included on the ground station. Since the ground station's mount will not be moving, only a low-accuracy accelerometer is assumed to be adequate for orientation. The \textit{MPU-9250} is therefore selected, as it is an inexpensive, readily-available option.

\subsection{RF Module}
The transciever type for both the custom and radiosonde link needs to be considered. There are a few options in this regard, listed from more to less specialized:
\begin{itemize}
    \item Fully custom design. Here, all components are discretely designed using transistors, MCUs etc. This option is listed for completeness, but is out of the scope of this project.
    \item Front-End Module (FEM). For this option, the FEM provides filters, a low-noise amplifier (LNA), antenna matching, and down-conversion. Then, a controller (e.g. FPGA or MCU) needs to be designed to perform modulation/demodulation functionality (the \textit{modem}) and interface with the FEM.
    \item Dedicated Transciever. A transciever provides both the FEM functionality and the software modulation, however still requires RF techniques for matching, as well as an additional RF shield.
    \item Dedicated module. For this option, all functionality is provided in a dedicated module, with a simple antenna connection/pin and MCU interface. This method has little flexibility, since all matching is done internally, and the frequency band is therefore constrained, however provides highest ease-of-use.
    \item Software-defined radio (SDR). This is the most general-purpose option, but does not exercise the highest performance. It is often directly connected to a computer.
\end{itemize}

\subsubsection{Custom Communication}
For the custom link, a dedicated LoRa module will be used. Most of these modules include GFSK as a built-in alternative, which makes them the ideal choice. All LoRa ICs are based on \textit{Semtech} chipsets due to the company's patent on the technology. Although dedicated transceivers are available, these require special RF considerations to be made, as discussed, and therefore a module will be utilized to ease development. Table \ref{tab:rfModules} contrasts the two most common modules in the 433 MHz LoRa band. The RA-02 is selected due to availability.

\begin{table}[!htb]
  \centering
  \renewcommand{\arraystretch}{1.2}
  \begin{tabular}{ |c|c|c|c| }
  \hline
  \textbf{Name}   & \textbf{RX Sensitivity} & \textbf{TX Sensitivity}& \textbf{Frequency} \\
  \hline
  RA-02           & -141 dBm             & +18 dBm              & 410 to 525 MHz     \\
  RFM98           & -148 dBm             & +20 dBm              & 410 to 525 MHz     \\
  \hline
  \end{tabular}
  \caption{Comparison of two LoRa modules}
  \label{tab:rfModules}
\end{table}

\subsubsection{Radiosonde Communication}
Unfortunately, although the selected LoRa module supports GFSK demodulation, it does not receive in the meteorological band, and therefore a second receiver is needed for the radiosonde link. In order to cater for a wide variety of radiosondes, a software-defined radio (SDR) can be used to adjust the signal decoding parameters dynamically. The \textit{RTL-SDR} USB dongle is considered the standard, low-cost solution for this, and is therefore selected. An alternative option is to design a dedicated GFSK receiver which allows for adjustable parameters, however this is not selected due to time constraints.

\subsection{Stepper Motor Driver}
The original PCB of the previous antenna mount contained two A3972 ICs to drive the stepper motors. Although these could be de-soldered and used, it is favourable to use a newer IC for easy replacement in case of damage. From the driver and motor datasheets, it is realized that the drivers should operate in dual DMOS full-bridge output configuration; allow bipolar PWM current control; allow for micro-stepping; and provide 0.5 A at 24 V. Very few drivers fit these exact specifications. The \textit{A4970} is considered the follow-up version of the \textit{A3972}, however is not available in a single pack or locally. The \textit{L6219} driver is found to match these specifications while only allowing half-stepping. It is therefore selected as it is the closest available option that matches these specifications.

\subsection{Microcontroller}
A general-purpose MCU will be needed both on the GS and the PQ. The three most popular frameworks/MCU types for this are the ESP32, Arduino, and STM32.

The \textit{ESP32-WROOM-32} is a commonly available ESP32 variant which has 32 GPIO pins, of which several can be used as an ADC input, SPI or I2C interface. This means that there are 23x GPIO available - enough for this project. Further, the ESP32 has the added benefit of WiFi and Bluetooth connectivity, which may allow “smart” features to be added to either devices in the future. The \textit{ATMEGA328PB} is an MCU commonly used in small Arduino boards. It, however, has fewer pins than the ESP, is an 8-bit MCU, and has only 32 KB of flash memory.. It does, however, allow a supply voltage of as low as 1.8V, and has very low current consumption. Finally, the common \textit{STM32F411} MCU is a high-performance, ARM-Cortex board advertised to have highly accurate ADCs and up to 81 GPIOs.

Since the processing requirements for this project are not very high, and the ESP32 is a relatively high-performance MCU which satisfies the pin requirements, with added "smart" capabilities, it will be selected for the GS. For the PQ unit, the ATMEGA328PB will be used, due to its exceptionally low current consumption, and many fewer pins being needed for the PQ unit.

\subsection{RS-485 Driver}
An RS-485 driver is needed for compatibility with the PocketQube interface. Due to the nature of the project, where bus communication happens between physically close units, only short-range RS-485 communication should be catered for. Further, the PQSU standard specifies no speed requirements, and therefore a minimum of 9600 baud will be designed for. The \textit{MAX485} is selected due to its availability, and the fact that it meets these specifications.

\subsection{Power}
Both the ground station and the PocketQube should be designed such that they can operate for a full three-hour flight. The total power calculations for the ground station and the PocketQube unit are listed in the Appendix in Tables \ref{tab:gs_component_consumption} and \ref{tab:pqunit_component_consumption}, resulting in 50.0 W and 453 mW respectively. It is decided to use a 12 V lead-acid battery for the ground station, and a 3.7 V LiPo battery for the PocketQube unit.

For the ground station, the battery should be capable of supplying around 4 A for the full flight, leading to a minimum capacity requirement of 12 Ah. It is decided that a more readily available 7.2 Ah battery will be used for development, and a larger battery purchased if the full flight takes place. Linear regulators will be used for the 3.3 V and 5 V supply (as opposed to switched-mode) both due to their simplicity, and the lack of any system-wide efficiency specifications. The 3.3 V regulator should be capable of supplying around 130 mA to the RF module, GPS, and IMU, and the 5 V regulator around $\SI{120}{mA} + \SI{130}{mA} = \SI{250}{mA}$ for the other components. The LD1117CV and L7805CP are readily-available linear regulators which can supply up to $\SI{800}{mA}$ and $\SI{1.5}{A}$ respectively, and therefore meet the requirements. A boost converter will be used to achieve the 24 V motor drive voltage. Since each stepper motor will draw around 0.5 A per coil, the converter should support 2 A at its 12 V input. A breakout board based on the XL6009 IC will be used, which meets this requirement.

For the PocketQube unit, the LiPo battery should be capable of supplying around 120 mA nominally. To achieve the 3 hour specification, a capacity of 360 mAh is therefore required. A larger 2000 mAh battery will be used for development, due to its low-cost; the added convenience of fewer discharge cycles; as well as the allowance it provides for retrieving the payload after the balloon has landed at the end of the day ($\approx$ 16 hours later).