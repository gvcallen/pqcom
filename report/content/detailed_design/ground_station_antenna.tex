\graphicspath{{./figures}}

\section{Ground Station Antenna}
\subsection{Mechanical Considerations}

The existing antennna mount is shown in Figure \ref{fig:antennaMount}. Azimuthal and elevation stepper motors are connected directly through Gear A and B respectively. Gear A rotates the centre shaft to provide azimuthal steering, whereas Gear B rotates through Gear D and E to allow a change in elevation (pointing angle). The 
specifications in Table \ref{tab:mount_specifications} were realized.

\begin{table}[!htb]
  \centering
  \renewcommand{\arraystretch}{1.2}
  \begin{tabular}{ |c|c| }
  \hline
  \textbf{Component}        & \textbf{Specification}    \\
  \hline
  Motor         & 200 full steps (per 360 degrees) \\ \hline
  Gear A        & 15 teeth \\ \hline
  Gear B        & 20 teeth \\ \hline
  Gear C        & 60 teeth \\ \hline
  Gear D        & 92 inner teeth, 80 outer teeth \\ \hline
  Gear E        & 140 teeth (equivalent) \\ \hline
  \end{tabular}
  \caption{Mount Component Specifications}
  \label{tab:mount_specifications}
\end{table}

The black plastic platform is triangular and has three mounting holes. The new ground station's antenna should therefore be mounted onto this platform. It is important to consider the forces/torques involved, as well as the stepper motor holding torques, in order to determine the maximum weight and acceptable form factor of the new antenna.

The azimuthal motor need not be considered, as all torques from the mount to its base are perpendicular to the central shaft, meaning it would only need to overcome static frictional forces to rotate the antenna, which are assumed to be negligible. The holding torque of the elevation motor, however, will constrain the antenna design. The holding torque of each motor is 0.25 Nm. The gearing ratio resulting from Gears B, D and E, results in an around 8x torque increase. This provides a holding torque of around 2 Nm at the pivot axle.

The centre of gravity of the antenna, as well as its weight, will affect the motor's ability to hold it in place. The horizontal distance between the pivot axle and the mount in the upright position is measured to be 40mm. Therefore, in the worst-case (when the mount is upright as in Figure \ref{fig:antennaMount}) a planar antenna could weigh up to $\frac{2}{(9.8 \times 0.04)} = \SI{5.1}{kg}$. In general, the "mass-distance" product of the antenna (mass times distance of centre of gravity from mount) should not be more than $\SI{0.2}{kg \cdot m}$. This should be strongly considered if a longer antenna is desired.
