\graphicspath{{./figures}}

\section{Ground Station PCB}
\subsection{Circuit Design}
For the ground station schematic, ``typical application" circuits found in component datasheets were mostly used. The final schematic can be found in Appendix \ref{sec:appendix_gs_schematics}.

\subsubsection{Motor Drivers}
The L6219 drivers require a reference voltage to set the maximum current. According to the datasheet \cite{datasheet-L6219}, a peak current of $I_{max} = 10 \times \frac{V_{ref}}{R_s}$ in mA is used. Since the recommended value of $R_s = \SI{1}{\ohm}$ is being used, $I_{max} = 10 \times V_{ref}$ mA. Since the motors allow up to 500 mA per winding, a current value of $I_{max} = \SI{475}{mA}$ is chosen to prevent damage. Therefore, $V_{ref} = \SI{4.75}{V}$, which can be implemented with a voltage divider. Setting the upper resistor $R_1 = \SI{1}{\kilo \ohm}$ results in $R_2 \approx \SI{20}{\kilo \ohm}$, as shown in the schematic. The disadvantage of this method is that the maximum current cannot be controlled dynamically. The reference can be replaced with a DAC from the MCU if more control is required.

\subsubsection{Connectors}
The RA-02 RF module states that not all data pins are necessary. Further, it was assumed that there may be some errors in the original design. A male header was therefore provided for unused GPIO pins, to allow for flexibility once the PCB was manufactured.

\subsection{PCB Layout}
A 2-layer stackup was decided on, since the in-house PCB machine has a 2-layer limitation, and it is desireable to manufacture the board as early as possible. The final PCB was designed in KiCAD. The layout is shown in Appendix \ref{sec:appendix_gs_pcb_design}. Female headers were used for most of the modules, since development boards were being used. Two PCB pours were exposed below the voltage regulators to act as a heat sink for added dissipation.