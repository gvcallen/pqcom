\section{Link Budget}

For any communication system, a link-level design or \textit{link budget} should be conducted. The goal of this step is to determine the power requirements of the commuication system, taking into account transmitter power, antenna characteristics, receiver sensitivity, and any attenuation involved.

Given the travel path in Figure \ref{fig:balloon_path}, and a height of 30 km, the minimum and maximum elevation angles are around 15 and 17 degrees respectively.

**Transmitter Power**
+18 dBm (RA-02)

**GS Gain**
+1.2 dBi

**Satellite Gain**
+1.2 dBi + G (if directive antenna is used)

**Receiver Sensitivity**
Target of around 9600 bps:
    +118 dBm (LoRa, SF 6, 125 kHz, CR2, 7812 bps)
    +120 dBm (LoRa, SF 7, 250 kHz, CR2, 9115 bps)
    +119 dBm (LoRa, SF 8, 500 kHz, CR2, 10417 bps)
Target of maximum sensitivity:
    +139 dBm (LoRa, SF 12, 62.5 kHz, CR1, 146 bps)
    +147 dBm (LoRa, SF 12, 62.5 kHz, CR1, 24 bps)
Target of maximum speed:
    +116 dBm (LoRa, SF 7, 500 kHz, CR2, 18230 bps) (high)
    +111 dBm (LoRa, SF 6, 500 kHz, CR2, 37500 bps) (very high)
    +109 dBm (GFSK 40 kHz, 38400 bps) (very high)
    +108 dBm (OOK, 32000 bps) (very high)

**Free space loss**
-133 dB (250 km)
-153 dB (2500 km)

**Polarization mismatch**
-3 dB

**Helical parameters**
Disc diameter = 762
N               Gain (dBi)          Length (mm)
-----------------------------------------------------
3               10.55               520
4               11.80               693
5               12.77               866
8               14.81               1385
10              15.78               1732
15              17.54               2598

**Budgets**
Range           Speed               Budget
---------------------------------------------------------------------------------------------------
Close           Medium              18dBm + 1.2dBi + 1.2dBi + 119dBm - 133dB - 3dB = +3.4dB margin
Close           High                18dBm + 1.2dBi + 1.2dBi + 116dBm - 133dB - 3dB = +0.4dB margin
Close           Very high           18dBm + 1.2dBi + 1.2dBi + 108dBm - 133dB - 3dB = -7.6dB short
Far             Low                 18dBm + 1.2dBi + 1.2dBi + 139dBm - 153dB - 3dB = +3.4dB margin
Far             Medium              18dBm + 1.2dBi + 1.2dBi + 119dBm - 153dB - 3dB = -16.6dB short
Far             High                18dBm + 1.2dBi + 1.2dBi + 116dBm - 153dB - 3dB = -19.6dB short
Far             Very high           18dBm + 1.2dBi + 1.2dBi + 108dBm - 153dB - 3dB = -27.6dB short

Therefore:
- A simple half-wave dipole will work for close-range medium-speed LoRa, and far-range high-speed LoRa
- Around 10 dBi antenna gain is needed for high-speed close-range applications (leaving 2.4 dB margin)
- Around 20 dBi antenna gain is needed for medium-speed far-range applications (leaving 2.4 dB margin)