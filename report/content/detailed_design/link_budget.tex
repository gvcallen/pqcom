\section{Link Budget}

The link budget should now be conducted, as well as initial considerations for the ground station geo-location. This location will affect the range of elevation angles to accomodate, the distance to the horizon, as well as the free-space path loss.

If it is acceptable to estabilish the link a few minutes after the launch, Signal Hill in Cape Town is a viable location for the ground station, and is easily accessible. At its height of 350 m, communication with the balloon would move be available after the ballon reaches an altitude of 650 m. A maximum elevation angle of $13.5 \circ$ (to Worcester) should therefore be considered.

If communication with the satellite from the beginning of launch is necessary (e.g. for direct GPS tracking), then the ground station would need to be raised above sea level. Considering Cape Town as a location would require a height of around 1 km (measured to Saldanha Airfield), which is mostly impractical, unless accessibility into Table Mountain is granted. Therefore, a location scout would need to be conducted closer to the launch site. At 200 m altitude, the site could be up to 50 km away from launch. In this case, a maximum elevation angle of between $25 \circ$ and $30 \circ$ will need to be considered.

The \textit{receiver sensitivity} is a value which denotes the minimum power level required at the receiver for successful communication. Depending on set parameters, the SX1278 quotes varying receiver sensitivity values. Notable configurations are listed in Table \ref{tab:sensitivity_values}.
\begin{table}[!htb]
  \centering
  \renewcommand{\arraystretch}{1.2}
  \begin{tabular}{ |c|c|c|c|c| }
  \hline
  \textbf{Description} & \textbf{Modulation} & \textbf{Parameters} & \textbf{Bit Rate} & \textbf{Sensitivity} \\
  \hline
  Maximum Bit Rate &
  GFSK &
  62.5 kHz &
  250 000 bps &
  -92 dBm \\
  \hline
  High Bit Rate &
  GFSK &
  40 kHz &
  38400 bps &
  -109 dBm \\
  \hline
  Target Bit Rate ($\approx 9600 bps$) &
  GFSK &
  5 kHz &
  4800 bps &
  -115 dBm \\
  \hline
  - &
  LoRa &
  500 kHz, SF 8 &
  10417 bps &
  -119 dBm \\
  \hline
  Maximum Sensitivity &
  LoRa &
  62.5 kHz, SF 12 &
  24 bps &
  -147 dBm \\
  \hline
  \end{tabular}
  \caption{Notable SX1278 Configurations}
  \label{tab:sensitivity_values}
\end{table}

The software mentioned in Section \ref{sec:link_budget} was used to compute attenuation effects in the link budget, with Saldanha Airfield as the satellite location, and Signal Hill as the ground station location.
All calculations were based on the RA-02's internal SX1278 chip and its specifications, half-wavelength dipoles as antennas, and worst-case impedance mismatch loss of 1.2 dBi (a VSWR 3:1, taken from the SX1278 datasheet). Then, the resultant margin will be analysed, and an appropriate directive antenna will be designed for the ground station to allow for increased margin and performance. The LoRa "target bit rate" sensitivity value of -119 dBm is used in this budget.

\newpage

\begin{table}[!htb]
  \centering
  \renewcommand{\arraystretch}{1.2}
  \begin{tabular}{ |c|c| }
  \hline
  Transmitter Power             & 17 dBm                    \\ \hline
  Mismatch Loss                 & (1.2 dBi)                 \\ \hline
  Transmission Line Loss        & (0.1 dB)                  \\ \hline
  Antenna Gain                  & 2.15 dB                   \\ \hline \hline
  \textbf{EIRP}                 & \textbf{17.85 dBm}        \\ \hline
  \end{tabular}
  \caption{Link Budget - Satellite}
  \label{tab:link_budget_satellite}
\end{table}

\begin{table}[!htb]
  \centering
  \renewcommand{\arraystretch}{1.2}
  \begin{tabular}{ |c|c| }
  \hline
  Free space path loss (110 km) & 125.9 dB                  \\ \hline
  Atmospheric loss              & 0.22 dB                   \\ \hline
  Scintillation loss            & 0.37 dB                   \\ \hline
  Polarization Mismatch         & 3 dB                      \\ \hline
  \textbf{Total Loss}           & \textbf{129.49 dB}         \\ \hline
  \end{tabular}
  \caption{Link Budget - Channel}
  \label{tab:link_budget_channel}
\end{table}

\begin{table}[!htb]
  \centering
  \renewcommand{\arraystretch}{1.2}
  \begin{tabular}{ |c|c| }
  \hline
  Antenna Gain                  & 2.15 dBi        \\ \hline
  Transmission Line Loss        & (0.1 dB)        \\ \hline
  Mismatch Loss                 & (1.2 dB)        \\ \hline
  Receiver Sensitivity          & 119 dBm         \\ \hline
  \textbf{GS Sensitivity}       & 119.85 dBm      \\ \hline
  \end{tabular}
  \caption{Link Budget - Ground Station}
  \label{tab:link_budget_gs}
\end{table}

The final link budget therefore results in 17.85 dBm - 129.49 dB + 119.85 dBm = 8.21 dB of link margin. Although this is theoretically adequate, there are a few considerations to be made:
\begin{itemize}
    \item The application note in \cite{paper-linkBudget} suggests a minimum link margin of 10 dB. It also further suggests a margin of above 20 dB for critical links.
    \item If a higher bit rate is required, e.g. 38.4 kbps, then the margin drops to -1.79 dB.
    \item If a longer range (e.g. for future projects) is required, free space path loss will drastically increase. For example, for a 250 km range, the loss becomes 133 dB, meaning the margin drops to only 1.11 dB.
\end{itemize}

To achieve the 10 dB recommended margin, an antenna with at least 1.79 dB additional gain (i.e. 3.94 dBi total) should be designed for the ground station. Then, if necessary, the 20 dB recommendation can be reached by lowering the bit rate.