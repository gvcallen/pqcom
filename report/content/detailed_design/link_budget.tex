\section{Link Budget}

The link budget should now be conducted, as well as initial considerations for the ground station geo-location. This location will affect the range of elevation angles to accomodate, the distance to the horizon, as well as the free-space path loss.

If it is acceptable to estabilish the link a few minutes after the launch, Signal Hill in Cape Town is a viable location for the ground station, and is easily accessible. At its height of 350 m, communication with the balloon would move be available after the ballon reaches an altitude of 650 m. A maximum elevation angle of $13.5 \circ$ (to Worcester) should therefore be considered.

If communication with the satellite from the beginning of launch is necessary (e.g. for direct GPS tracking), then the ground station would need to be raised above sea level. Considering Cape Town as a location would require a height of around 1 km (measured to Saldanha Airfield), which is mostly impractical, unless accessibility into Table Mountain is granted. Therefore, a location scout would need to be conducted closer to the launch site. At 200 m altitude, the site could be up to 50 km away from launch. In this case, a maximum elevation angle of between $25 \circ$ and $30 \circ$ will need to be considered.

The software mentioned in Section \ref{sec:link_budget} was used to compute attenuation effects in the link budget, with Saldanha Airfield as the satellite location, and Signal Hill as the ground station location. 

**Transmitter Power**
+18 dBm (RA-02)

**GS Gain**
+1.2 dBi

**Satellite Gain**
+1.2 dBi + G (if directive antenna is used)

**Receiver Sensitivity**
Target of around 9600 bps:
    +118 dBm (LoRa, SF 6, 125 kHz, CR2, 7812 bps)
    +120 dBm (LoRa, SF 7, 250 kHz, CR2, 9115 bps)
    +119 dBm (LoRa, SF 8, 500 kHz, CR2, 10417 bps)
Target of maximum sensitivity:
    +139 dBm (LoRa, SF 12, 62.5 kHz, CR1, 146 bps)
    +147 dBm (LoRa, SF 12, 62.5 kHz, CR1, 24 bps)
Target of maximum speed:
    +116 dBm (LoRa, SF 7, 500 kHz, CR2, 18230 bps) (high)
    +111 dBm (LoRa, SF 6, 500 kHz, CR2, 37500 bps) (very high)
    +109 dBm (GFSK 40 kHz, 38400 bps) (very high)
    +108 dBm (OOK, 32000 bps) (very high)

**Free space loss**
-133 dB (250 km)
-153 dB (2500 km)

**Polarization mismatch**
-3 dB

**Helical parameters**
Disc diameter = 762
N               Gain (dBi)          Length (mm)
-----------------------------------------------------
3               10.55               520
4               11.80               693
5               12.77               866
8               14.81               1385
10              15.78               1732
15              17.54               2598

**Budgets**
Range           Speed               Budget
---------------------------------------------------------------------------------------------------
Close           Medium              18dBm + 1.2dBi + 1.2dBi + 119dBm - 133dB - 3dB = +3.4dB margin
Close           High                18dBm + 1.2dBi + 1.2dBi + 116dBm - 133dB - 3dB = +0.4dB margin
Close           Very high           18dBm + 1.2dBi + 1.2dBi + 108dBm - 133dB - 3dB = -7.6dB short
Far             Low                 18dBm + 1.2dBi + 1.2dBi + 139dBm - 153dB - 3dB = +3.4dB margin
Far             Medium              18dBm + 1.2dBi + 1.2dBi + 119dBm - 153dB - 3dB = -16.6dB short
Far             High                18dBm + 1.2dBi + 1.2dBi + 116dBm - 153dB - 3dB = -19.6dB short
Far             Very high           18dBm + 1.2dBi + 1.2dBi + 108dBm - 153dB - 3dB = -27.6dB short

Therefore:
- A simple half-wave dipole will work for close-range medium-speed LoRa, and far-range high-speed LoRa
- Around 10 dBi antenna gain is needed for high-speed close-range applications (leaving 2.4 dB margin)
- Around 20 dBi antenna gain is needed for medium-speed far-range applications (leaving 2.4 dB margin)