\graphicspath{{./figures}}

\section{Radiosonde}

The helical antenna was setup to receive the radiosonde GFSK signal using the RTL-SDR USB dongle and an SMA to MCX adapter. A resultant "squelch" signal was observed to be received in one second intervals, and the frequency domain of this signal is plotted in Figure \ref{fig:radiosondeSpectrum} using GNU Radio software. A free Python script \textit{Auto-RX} was found to cater for receiving telemetry from the iMet-54 Radiosonde, and the GPS location was successfully received. The developed GUI was therefore altered to allow for tracking the external Radiosondes, and the closed-loop tracking was successfully tested with a similiar but shorter-range procedure as in \ref{sec:closed_loop_testing}.

\begin{figure}[!htb]
  \centering
  \includegraphics[width=0.8\textwidth]{radiosondeSpectrum}
  \caption{Received Radiosonde Signal FFT}
  \label{fig:radiosondeSpectrum}
\end{figure}