\section{Requirements}

From the information listed in the problem statement, and after supervisor consultation, slightly more specific user requirements are gathered and listed:
\begin{enumerate}
    \item The GS should be capable of receiving continuous data wirelessly from the PQ and the radiosonde. The PQ should be theoretically capable of responding to commands from the GS.
    \item The communication link should be maintained throughout the predicted path covered in Figure \ref{fig:balloon_path}
    \item The PQ unit should remain operational for at least the launch time of the predicted path, which is 2 hours 25 minutes. To cater for longer flights, a goal of 3 hours is set.
    \item The PQ unit should conform to the \textit{SU-modified} PQ9 standard (listed in Appendix \ref{sec:appendix_pqsu}), and integrate with other prototype units. This standard is here-on referred to as \textit{PQSU}.
    \item The GS electronics should integrate into an existing antenna mount, which has been provided by the E\&E Department.
\end{enumerate}

The system will undergo a \textit{flight-readiness review} to determine if it has met the requirements before launch. Although the GS could be placed closer to the launch location if the range requirement is not met (while affecting the communication time) this should only be used as a last resort. Further, as the project progresses, if supporting greater distances ($> 200$ km) will not significantly increase the time, complexity, or cost of the system, it could be catered for as an expansion to the core requirements. It should be noted, however, that the nature of this project (a system prototype) means that such optimisations should not be made a priority.