\section{Requirements}

As mentioned, the general requirement of this system is that continuous, wireless communication should be established and maintained between the ground station and a balloon-satellite carrying a PocketQube payload. From this information, and after supervisor consultation, slightly more specific user requirements are gathered and listed:
\begin{enumerate}
    \item The GS should be capable of receiving data continuously and wirelessly from the PQ, as well as data from the priorietary Radiosonde.
    \item The communication system should be capable of the range covered in Figure \ref{fig:balloon_path}.
    \item The PQ unit should conform to the \textit{SU-modified} PQ9 standard (listed in Appendix \ref{sec:appendix_pqsu}), and integrate with other prototype units. This standard is here-on refered to as \textit{PQSU}.
    \item The PQ unit should remain operational for "a few hours".
    \item The GS electronics should integrate into an existing antenna mount.
\end{enumerate}

The system will undergo a \textit{flight-readiness review} to determine if it has met the requirements before launch. These requirements mostly include the neccessity to monitor the satellite from take-off and for an hour thereafter, and in that case the full 110 km range would not be required as the ground station could theoretically be placed closer. However, the system should be designed for this range, in order to cater for the full flight path. Further, as the project progresses, if supporting greater distances (200+ km) will not significantly increase the time, complexity, or cost of the system, it could be catered for as an expansion to the core requirements. It should be noted, however, that the nature of this project (a system's level design) means that such optimizations should not be made a priority.