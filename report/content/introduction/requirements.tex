\section{Requirements}

As mentioned, the general requirement of this system is that continuous, wireless communication should be established and maintained between the ground station and a balloon-satellite carrying a PocketQube payload. From this information, and after supervisor consultation, slightly more specific user requirements are gathered and listed:
\begin{enumerate}
    \item The GS should be capable of receiving data continuously and wirelessly from the PQ, as well as data from the Radiosonde.
    \item The communication system should be capable of the range covered in Figure \ref{fig:balloon_path}.
    \item The PQ unit should conform to the \textit{SU-modified} PQ9 standard (listed in Appendix \ref{sec:appendix_pqsu}), and integrate with other prototype units. This standard is here-on referred to as \textit{PQSU}.
    \item The PQ unit should remain operational for "a few hours".
    \item The GS electronics should integrate into an existing antenna mount.
\end{enumerate}

The system will undergo a \textit{flight-readiness review} to determine if it has met the requirements before launch. Although the GS could be placed closed with a decreased communication time, the system should be designed for the above range, in order to cater for the full flight path. Further, as the project progresses, if supporting greater distances (200+ km) will not significantly increase the time, complexity, or cost of the system, it could be catered for as an expansion to the core requirements. It should be noted, however, that the nature of this project (a system's level design one) means that such optimisations should not be made a priority.