\graphicspath{{./figures}}

\section{Specifications}

In order to define a more detailed list of specifications, further calculations should be done. Balloon satellites rise at a vertical speed of around 20 km/h, and safely fall at around 35 km/h after bursting if a parachute is used \cite{site-weatherWeatherBalloons}. The predicted flight path in Figure \ref{fig:balloon_path} shows a travelled horizontal distance of about 60 km in 40 minutes after the balloon bursts, resulting in a horizontal speed of 90 km/h. A speed of around 100 km/h will be designed for, allowing slightly faster fall speeds.

At a maximum height of 30 km, a calculator reveals that the horizon is around 600 km \cite{site-normalHorizon}. The GS antenna could therefore be placed at sea level if communication is only required near maximum altitude. If the balloon should instead be tracked within one minute after launch (300 metres), then a radar horizon calculator reveals that the GS should be placed at an altitude no lower than 200 metres \cite{site-radarHorizon}, assuming no refraction.

Again referring to the predicted path, a maximum slant range of just over 115 km is obtained, noting that the balloon moves closer as it increases in altitude. For simplification, if the maximum speed (28 m/s) is assumed to occur perpendicular to the ground station at this maximum range, a turn-rate requirement of $t = \frac{v}{r} = \frac{28}{115000} \approx 0.015^\circ / s$ is obtained.

The PQSU standard includes 3.3 V and 5 V bus voltages. Since this is a prototype launch, there is risk of other units (e.g. the EPS) malfunctioning. A power connection is also needed for development. A simple battery system which matches the PQSU voltage should therefore be included, to be used both for testing, and potentially for deployment.

Both systems will require PCBs which are constrained in size. PQSU defines the dimensions of a PocketQube PCB i.e. 42 mm x 42 mm outer dimensions, whereas the ground station PCB is constrained to fit onto the provided mount with similar sizing as the previous PCB that was used with it, as shown in Appendix \ref{sec:appendix_gs_pcb_existing}. Lastly, the system should drive the stepper motors which are already provided with the antenna mount, and may make use of the existing zero-sensor. The following system-level specifications are therefore drawn up:
\begin{enumerate}
    \item The system should be capable of a slant range of 120 km.
    \item The GS should be capable of tracking at a turning rate of $0.015^\circ / s$, and the feasibility of various tracking methods (e.g. open vs closed loop) should be explored.
    \item The system should be designed to operate at a minimum baud rate of 4800 (which a typical radiosonde uses \cite{datasheet-iMet54}) and a target baud rate of 9600 (which is a typical satellite telemetry downlink speed \cite{paper-deployableAntenna}).
    \item The system should allow for radiosonde data to be retrieved in the meteorological band, between 400.05 and 406 MHz \cite{datasheet-iMet54}. This data is generally GFSK modulated.
    \item A single antenna should be used for both the custom and radiosonde link on the GS, to simplify the design. This antenna should therefore have a bandwidth in the range from 405 MHz up to the amateur radio band (433 MHz).
    \item A 100 mW equivalent transmit power restriction should be adhered to.
    \item The PQ unit should follow the PQSU and PQ9 standards, which stipulates: a 42 mm square outer PCB dimension; a 4 mm and 2 mm component height above and below respectively; and a 20-pin header interface, catering for RS-485 and I2C communication, and providing 3.3 V and 5 V power lines.
    \item The PQ unit should include a battery capable of lasting a minimum of 3 hours at nominal current draw.
    \item The GS PCB should integrate onto the existing antenna mount, which has a diameter of 198 mm, with equispaced mounting holes of 3 mm diameter.
    \item The GS should be mechanically steered using 4218S-15 bipolar stepper motors, which have a maximum current of 0.50 A per phase, and are driven at 24 V.
    \item The GS should provide a USB-C connection to allow a PC to monitor the telemetry data. This should be capable of receiving all data from the link in real-time.
\end{enumerate}