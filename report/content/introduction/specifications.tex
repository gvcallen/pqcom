\graphicspath{{./figures}}

\section{Specifications}

For a more detailed list of specifications to be defined, further calculations should be done. First, the required communication conditions (link distance, atmospheric effects etc.) are established. Then, power and voltage requirements are expanded on. Finally, system integration is considered.

Balloon satellites rise at a vertical speed of around 20 km/h \cite{site-weatherWeatherBalloons}, and can travel horizontally as fast as 200 km/h when falling. An average speed of around 100 km/h will be designed for, resulting in a flight time of longer than 2 hours. At a height of 30 km, a line-of-sight (LOS) calculator reveals that the horizon is around 600 km, meaning that the antenna could theoretically be placed at sea level if communication is only required near maximum altitude. However, the difference between the ground station and balloon height should not be less than 1 km, if communication from launch is desired. At these distances, the earth's curvature is found to be negligible, which results in a LOS distance of 120 km.

The PQSU standard includes 3.3V and 5V bus voltages. Since this is a prototype launch, there is risk of other units (e.g. the EPS) malfunctioning. A power connection is also needed for development, and therefore a simple battery system which matches the standard's voltage should be included in the design to be used for testing and potentially for deployment.

The standard also defines the dimensions of a PocketQube PCB (e.g. 42 mm x 42 mm outer dimensions). The ground station PCB dimensions are constrained by to be similar to the existing circular PCB, which contains mounting holes and two support "wings" as shown in Appendix \ref{sec:appendix_gs_pcb_existing}. Lastly, the system should drive the stepper motors which are already provided with the antenna mount, and may make use of the existing zero-sensor. The following system-level specifications are therefore drawn up:
\begin{enumerate}
    \item The system should be capable of a slant range of 120 km.
    \item The system should be designed to operate at a minimum baud rate of 4800 (which the iMet-54 uses \cite{datasheet-iMet54}) and a target baud rate of 9600 (which is a typical satellite telemetry downlink speed as in \cite{paper-deployableAntenna}).
    \item The system should allow for iMet-54 radiosonde data to be retrieved. This data is GFSK modulated at a pre-selected frequency of between 402 to 405 MHz \cite{datasheet-iMet54}.
    \item A single antenna should be used for both the custom and radiosonde link on the GS, to simplify the design. This antenna should therefore have a bandwidth in the range from 405 MHz up to the amateur radio band (433 MHz).
    \item A 100 mW equivalent transmit power restriction should be adhered to.
    \item The PQ unit should follow the PQSU and PQ9 standards, which stipulates: a 42 mm square outer PCB dimension; a 4 mm and 2 mm component height above and below respectively; and a 20-pin header interface, catering for RS-485 and I2C communication, and providing 3.3 V and 5 V power lines.
    \item The PQ unit should include a battery capable of lasting 2 hours at nominal current draw.
    \item The GS should be capable of tracking the balloon at 100 km/h, and at a line-of-sight distance of 120 km.
    \item The GS PCB should integrate onto the existing antenna mount, which has a diameter of 198 mm, with equispaced mounting holes etc. (as in \ref{fig:gs_existing}).
    \item The GS should control two 4218S-15 bipolar stepper motors, which have a maximum current of 0.50 A per phase, and are driven at 24 V.
    \item The GS should provide a USB-C connection to allow a PC to monitor the telemetry data. This should be capable of receiving all data from the link in real-time.
\end{enumerate}