\section{Requirements}

The general requirement of the system is that continuous, wireless communication should be established and maintained between the ground station and a balloon-satellite carrying a PocketQube payload. The antenna mount of an existing ground station has been provided. Further, communication with the existing Radiosonde should also be possible. From this information, and after supervisor consultation, slightly more specific user requirements are gathered and listed:
\begin{enumerate}
    \item Wireless communication data between the GS and the PQ should be retrievable, as well as data from the priorietary Radiosonde.
    \item The communication system should be capable of the range covered in Figure \ref{fig:balloon_path}.
    \item The PQ unit should conform to the \textit{SU-modified} PQ9 standard (which has been provided), and integrate with other prototype units. This standard is hereby refered to as PQSU.
    \item The PQ unit should remain operational for "a few hours".
    \item The GS electronics should integrate into an existing antenna mount.
\end{enumerate}

The system will undergo a \textit{flight-readiness review} to determine if it has met the requirements before launch. To match the features of the existing, proprietary Radiosonde system \cite{datasheet-iMet3100M}, an additional requirement that the slant range should be no less than 250 km will be imposed. Further, although these are the minimum requirements for the Saldanha launch, choices could be made such that the communication system is even more general-purpose, and can potentially be used for low earth orbit (LEO) applications as well. In this case, the orbital height should be increased to between 160 km and 1000 km \cite{site-esaTypesOrbits}. As the project progresses, if supporting such a distance will not significantly increase the time, complexity, or cost of the system, it could be catered for as an expansion to the core requirements.