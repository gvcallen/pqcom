\section{Future Work}\label{sec:future_work}

There are various improvements and alternative choices that can be explored in the context of the developed system:
\begin{itemize}
    \item \textit{Improved ground station PCB}. The ground station PCB was designed to use break-out boards, through-hole components, and modules for rapid development. Future designs could replace almost all components with surface-mounted ones, and remove break-out boards (e.g. the ESP32 board) to reduce cost, improve connection integrity, and allow more space for additioanl components.
    \item \textit{Improved helical antenna construction}. The ground station antenna can be optimised, by constructing it using more rigid materials, as well as stricter manufacturing procedures. This would hopefully decrease the variance between the simulation and the implementation, and improve the matching and radiation pattern. A longer antenna with higher gain could also potentially be implemented if similar materials are used, considering that the mass-distance product of the implemented antenna was under half of the theoretical maximum.
    \item \textit{PocketQube antenna integration}. The antenna for the PocketQube was not designed to be deployable, but the half-dipole was shown to provide an effective proof-of-concept. For integration into future PocketQubes, options include using a higher frequency band; making the antenna foldable with a spring mechanism; or using a much lower efficiency normal-mode helical antenna.
    \item \textit{Radiosonde link optimisation}. The radiosonde link and integration was not fully explored. For example, a dedicated, more sensitive transceiver targeted at the relevant frequency (as opposed to a generic software-defined radio) could be explored. This might require an RF switch, which would provide the ability to change between the custom and radiosonde link in software.
    \item \textit{Signal strength scanning}. Faster signal strength "conical" scanning could be explored. Although it was not investigated in this project (as a brute-force scan or open-loop estimate is generally sufficient to initially find the payload and receive its GPS location) the flexibility of tracking the payload using only its received radio signal may be beneficial.
\end{itemize}