\section{Future Work}\label{sec:future_work}

There are various improvements and alternative choices that could be explored in the context of the developed system:
\begin{itemize}
    \item \textit{Improved ground station PCB}. The ground station PCB was designed to use break-out boards, through-hole components, and modules for rapid development. Future designs could replace almost all components with surface-mounted ones, and remove break-out boards (e.g. the ESP32 board) to reduce cost, improve connection integrity, and allow more space for additional components.
    \item \textit{Improved helical antenna construction}. The ground station antenna can be optimised, by constructing it using more rigid materials to stricter manufacturing tolerances. This would hopefully decrease any variance between the simulation and the implementation, and improve the matching and radiation pattern. A longer antenna with higher gain could also potentially be implemented if similar materials are used, considering that the mass-distance product of the implemented antenna was under half of the theoretical maximum.
    \item \textit{Automatic orientation detection}. Currently, the ground station uses a dead-reckoning fix, requiring that it be pointed towards magnetic north, and that it is orientated flat. Further work could be done to make use of an IMU in order to allow the ground station to be rotated at any angle, with its orientation being automatically compensated for in software.
    \item \textit{PocketQube antenna integration}. The antenna for the PocketQube was not designed to be deployable, but the half-dipole was shown to provide an effective proof-of-concept. For integration into final PocketQube models, options include a deployable antenna (e.g. with a foldable spring mechanism); using a higher frequency band; or using a much smaller but lower efficiency normal-mode helical antenna.
    \item \textit{Radiosonde link optimisation}. A dedicated, more sensitive radiosonde receiver targeted at the meteorological band (as opposed to a generic software-defined radio) could be implemented. This might require an RF switch, but would provide the ability to change between the custom and radiosonde link in software.
    \item \textit{Signal strength scanning}. Faster signal strength ``conical" scanning could be explored. Although it was not investigated in this project (as a brute-force scan or open-loop estimate is generally sufficient to initially find the payload and receive its GPS location), the flexibility of tracking the payload using only its received radio signal may be beneficial.
    \item \textit{PocketQube assembly}. Since a prototype for all the sub-systems of a PocketQube communication system have been designed and tested, integration within an actual PocketQube housing can be investigated, and an actual PocketQube satellite can become a reality.
\end{itemize}