\section{Future Work}

There are various improvements and alternative choices that can be explored in the context of the developed system:
\begin{itemize}
    \item Improved helical antenna construction. The ground station antenna can be optimized by building it out of more rigid materials, as well as more strict manufacturing procedures, to meet decrease the variance between the simulation and the implementation, and improve the matching and antenna directivity.
    \item The antenna for the PocketQube was not designed to be deployable, but the half-dipole was shown to provide an effective proof-of-concept. This can worked on, to ensure the antenna is stowed during take-off, and realised in correct position during flight.
    \item The radiosonde protocol and integration was not fully explored. It would be very useful to be able to decode this information for future flights as a backup option. Further, a dedicated, more sensitive transceiver than an SDR at the relevant frequency, with an RF switch, could be explored. This would provide the ability to change between the custom and radiosonde protocol in software.
    \item Faster signal strength tracking could be explored further. Although it was not investigated, as a brute-force scan is generally sufficient to initially find the payload and record its GPS location, the flexibility of tracking the payload using only its RF signal may be beneficial.
\end{itemize}