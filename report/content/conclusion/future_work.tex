\section{Future Work}

There are various improvements and alternative choices that can be explored in the context of the developed system:
\begin{itemize}
    \item Improved helical antenna construction. The ground station antenna can be optimised, by constructing it using more rigid materials, as well as stricter manufacturing procedures. This would hopefully decrease the variance between the simulation and the implementation, and improve the matching and radiation pattern.
    \item The antenna for the PocketQube was not designed to be deployable, but the half-dipole was shown to provide an effective proof-of-concept. This can worked on, to ensure the antenna is stowed during take-off, and released in the correct position during flight.
    \item The radiosonde link and integration was not fully explored. It would be very useful to be able to decode this information for future flights as a backup option. Further, a dedicated, more sensitive transceiver targeted at the relevant frequency (as opposed to a generic software-defined radio) could be explored. This might include an RF switch, which would provide the ability to change between the custom and radiosonde link in software.
    \item Faster signal strength "conical" scanning could be explored. Although it was not investigated in this project (as a brute-force scan or open-loop estimate is generally sufficient to initially find the payload and receive its GPS location) the flexibility of tracking the payload using only its received radio signal may be beneficial.
\end{itemize}