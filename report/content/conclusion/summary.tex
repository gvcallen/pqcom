\section{Summary}

A PocketQube communication system, with a tracking ground station and a PocketQube PCB module, was successfully designed to meet the system requirements for a balloon satellite launch in Saldanha Bay. Unfortunately, the launch did not take place, however several modularized tests helped to prove that the system is fully functional and capable of communicating over the required distances.

LoRa proved to be a viable choice for the communication link. Not only did the link meet the throughput requirements, but the increased immunity due to the spread-spectrum technology suggests that it should be made a priority choice for satellite communication links, given that the required bandwidth is available. The helical antenna design also proved to be effective in the context of the project, as it was easy to manufacture and provided high gain and bandwidth for relatively low effort compared to alternatives.

The open-loop tracking method was successfully implemented, however would need to be tested in a full system test to confirm its accuracy. Closed-loop GPS tracking was shown to work over a short-range, and is likely the preferred tracking method, given that an initial communication link can be established with the satellite. Lastly, the final ground station and PocketQube module PCB designs and implementations were shown to be effective, and provide a good foundation for future work.
