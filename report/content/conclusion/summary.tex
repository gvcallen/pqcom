\section{Summary}

A PocketQube communication system, with a tracking ground station and a PocketQube PCB module, was successfully designed to meet the system requirements for a balloon satellite launch in Saldanha Bay. Unfortunately, the launch did not take place, however several modularized tests helped to prove that the system is fully functional and capable of communicating over the required distances.

LoRa proved to be a viable choice for the custom communication link. Not only did the link meet the throughput requirements of 9600 baud, but the system's range was successfully tested up to 122 km line-of-sight. Further, the increased immunity due to the spread-spectrum technology suggests that it should be made a priority choice for low-cost, low-power satellite communication links. The helical antenna design also proved to be effective in the context of the project, as it was easy to manufacture, and allowed the system to meet its requirements.

The open-loop tracking method was successfully implemented, however was not integrated into a full-system test. Closed-loop GPS tracking was successfully tested for a weather balloon launch up to a range of 25 km, and appeared to be the preferred tracking method for balloon satellite systems, since an initial GPS fix was easy to establish.

Lastly, the final ground station and PocketQube module PCB designs and implementations were shown to be effective, allowing a good foundation for future work.