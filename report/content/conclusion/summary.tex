\section{Summary}

A PocketQube communication system, with a tracking ground station and a PocketQube PCB module, was successfully designed to meet the system requirements for a balloon satellite launch in Saldanha Bay. Unfortunately, the launch did not take place, however several modularized tests helped to prove that the system is fully functional and capable of communicating of the required distances for the course of such a flight.

LoRa proved to be a viable choice for the communication link. Not only did the link meet the throughput requirements, but the increased immunity due to the spread-spectrum technology showed that it is is a priority choice for such sat-com links. The helical antenna design also proved to be effective in the context of the project, as it was easy to manufacture and provided high gain and bandwidth for relatively low effort compared to alternatives.

The open-loop tracking method was successfully implemented, however would need to be tested in a full system run. Closed-loop GPS tracking was shown to work over a short-range, and is claimed to be the prefered method if any sort of link can be estabilished with the satellite. Lastly, the final ground station and PocketQube module design were shown to be effective, and provide a good foundation for future work.
