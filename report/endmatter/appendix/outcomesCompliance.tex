% Outcomes Compliance
\chapter{Outcomes Compliance}
\begin{table}[!htb]
  \centering
  \renewcommand{\arraystretch}{1.2}
  \begin{tabular}{ |p{4cm}|p{8cm}|p{3cm}| }
  \hline
  \textbf{Outcome}        & \textbf{Description of Outcome in Report}     & \textbf{Chapter(s)} \\
  \hline
  \textbf{ELO 1. Problem Solving} &
  The project required developing a tracking satellite ground station with various constraints and requirements. The requirements and specifications had to be formulated in the project. The size and power constraints of both the ground station and the PocketQube unit had to be considered. The ground station mount had mechanical limitations, such as the physical size of the antenna that it could support, which presented a problem to be solved. &
  Chapters 1, 3 and 4 \\ \hline
  \textbf{ELO 2. Application of scientific and engineering knowledge} &
  The design of the ground station and antenna system incorporated engineering knowledge from various fields, such as electromagnetic theory; telecommunications theory; computer systems; computer programming; and controls systems. Finally, skills required for a higher-level systems design were also needed. &
  Chapters 3, 4, 5, 6 \\ \hline
  \textbf{ELO 3. Engineering Design} &
  The waterfall design approach was followed. First, a systems-level design was conducted to determine the required functionality of the system. Then, engineering design was done for each sub-system, which includes circuit and PCB design; antenna design and software design.  &
  Chapters 3 and 4 \\ \hline
  \textbf{ELO 4. Investigations, experiments and data analysis} &
  Hardware prototyping and testing was done on a breadboard to investigate the capabilities of the hardware, before implementing the final system. Various tests were then done on the final system, such as antenna measurements; long-range signal strength measurements; and GPS tracking tests. The results from these tests were analysed to determine if the system met the original requirements. &
  Chapters 5 and 6 \\ \hline
  \end{tabular}
  \caption{ECSA Outcomes Compliance 1}
  \label{tab:outcomesCompliance1}
\end{table}

\begin{table}[!htb]
  \centering
  \renewcommand{\arraystretch}{1.2}
  \begin{tabular}{ |p{4cm}|p{8cm}|p{3cm}| }
  \hline
  \textbf{Outcome}        & \textbf{Description of Outcome in Report}     & \textbf{Chapter(s)} \\
  \hline
  \textbf{ELO 5. Engineering methods, skills and tools, including Information Technology} &
  PCB Design software \textit{KiCAD} was used to design the PCBs. Electromagnetic CAD software \textit{FEKO} was used for the antenna design and simulation. The \textit{PlatformIO IDE} along with the \textit{Arduion Framework} and the \textit{C++} language were used to implement the software. Finally, \textit{Python} was used for data analysis. &
  Chapters 4, 5 and 6 \\ \hline
  \textbf{ELO 6. Professional and technical communication} &
  The final report is written in the appropriate format, structure, and formal language in order to communicate the report's findings. It was written in Latex, which is thoroughly used in professional and technical communication. &
   \\ \hline
  \textbf{ELO 8. Individual work} &
  The design, implentation and testing of the project was conducted entirely by the student, except for supervision from the lecturer. &
   \\ \hline
  \textbf{ELO 9. Independent Learning Ability} &
  The project requirements were drawn up independently given loose requirements. The antenna design techniques, tracking methodologies, simulation environments, and PCB design skills were indepently research throughout the course of the project. Further, the relevant mechanical knowledge, such as the stepper motor drive and the antenna manufacturing, was acquired independently through problem-solving and research. &
   \\ \hline
  \end{tabular}
  \caption{ECSA Outcomes Compliance 2}
  \label{tab:outcomesCompliance2}
\end{table}
\clearpage